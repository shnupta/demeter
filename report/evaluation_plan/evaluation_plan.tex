\chapter{Evaluation Plan}

As with the project plan, the evaluation will depend on the circumstances previously mentioned. 

I will first need to ensure the correctness of my implementations by comparing the outputs with existing tools/papers. The quantitative metrics collected by Savickas focus on the execution time of the different approaches, dependent on portfolio size, number of simulated paths and number of sensitivities calculated. Other key values will need to be obtained such as the error in any estimators.

In the fallback scenario it is likely I will be able to use \href{https://www.quantlib.org/}{QuantLib} to compare my implementations to standard CPU ones.

The main qualitative measure I will focus on will be the ease-of-implementation.